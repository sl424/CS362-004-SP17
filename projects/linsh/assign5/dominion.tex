%%%%%%%%%%%%%%%%%%%%%%%%%%%%%%%%%%%%%%%%%
% Programming/Coding Assignment
% LaTeX Template
%
% This template has been downloaded from:
% http://www.latextemplates.com
%
% Original author:
% Ted Pavlic (http://www.tedpavlic.com)
%
% Note:
% The \lipsum[#] commands throughout this template generate dummy text
% to fill the template out. These commands should all be removed when 
% writing assignment content.
%
% This template uses a Perl script as an example snippet of code, most other
% languages are also usable. Configure them in the "CODE INCLUSION 
% CONFIGURATION" section.
%
%%%%%%%%%%%%%%%%%%%%%%%%%%%%%%%%%%%%%%%%%

%----------------------------------------------------------------------------------------
%	PACKAGES AND OTHER DOCUMENT CONFIGURATIONS
%----------------------------------------------------------------------------------------

\documentclass{article}

\usepackage{fancyhdr} % Required for custom headers
\usepackage{lastpage} % Required to determine the last page for the footer
\usepackage{extramarks} % Required for headers and footers
\usepackage[usenames,dvipsnames]{color} % Required for custom colors
\usepackage{graphicx} % Required to insert images
\usepackage[rightcaption]{sidecap}
\graphicspath{ {images/} }
\usepackage{wrapfig}

\usepackage{listings} % Required for insertion of code
\usepackage{courier} % Required for the courier font
\usepackage{lipsum} % Used for inserting dummy 'Lorem ipsum' text into the template
\usepackage{comment}
\usepackage{amsmath,amsfonts,amsthm}

\lstset{%
   breaklines=true,
   extendedchars=true,
   basicstyle=\ttfamily,
   escapechar=~,
   literate={\→}{$\rightarrow$}1
}

% Margins
\topmargin=-0.45in
\evensidemargin=0in
\oddsidemargin=0in
\textwidth=6.5in
\textheight=9.0in
\headsep=0.25in

\linespread{1.1} % Line spacing

% Set up the header and footer
\pagestyle{fancy}
\lhead{\hmwkAuthorName} % Top left header
\chead{\hmwkClass\ : \hmwkTitle} % Top center head
\rhead{\firstxmark} % Top right header
\lfoot{\lastxmark} % Bottom left footer
\cfoot{} % Bottom center footer
\rfoot{Page\ \thepage\ of\ \protect\pageref{LastPage}} % Bottom right footer
\renewcommand\headrulewidth{0.4pt} % Size of the header rule
\renewcommand\footrulewidth{0.4pt} % Size of the footer rule

\setlength\parindent{0pt} % Removes all indentation from paragraphs

%----------------------------------------------------------------------------------------
%	CODE INCLUSION CONFIGURATION
%----------------------------------------------------------------------------------------

\definecolor{MyDarkGreen}{rgb}{0.0,0.4,0.0} % This is the color used for comments
\lstloadlanguages{Python} % Load Perl syntax for listings, for a list of other languages supported see: ftp://ftp.tex.ac.uk/tex-archive/macros/latex/contrib/listings/listings.pdf
\lstset{language=Python, % Use Perl in this example
        frame=single, % Single frame around code
        basicstyle=\small\ttfamily, % Use small true type font
        keywordstyle=[1]\color{Blue}\bf, % Perl functions bold and blue
        keywordstyle=[2]\color{Purple}, % Perl function arguments purple
        keywordstyle=[3]\color{Blue}\underbar, % Custom functions underlined and blue
        identifierstyle=, % Nothing special about identifiers
        commentstyle=\usefont{T1}{pcr}{m}{sl}\color{MyDarkGreen}\small, % Comments small dark green courier font
        stringstyle=\color{Purple}, % Strings are purple
        showstringspaces=false, % Don't put marks in string spaces
        tabsize=5, % 5 spaces per tab
        %
        % Put standard Perl functions not included in the default language here
        morekeywords={rand},
        %
        % Put Perl function parameters here
        morekeywords=[2]{on, off, interp},
        %
        % Put user defined functions here
        morekeywords=[3]{test},
       	%
        morecomment=[l][\color{Blue}]{...}, % Line continuation (...) like blue comment
        numbers=left, % Line numbers on left
        firstnumber=1, % Line numbers start with line 1
        numberstyle=\tiny\color{Blue}, % Line numbers are blue and small
        stepnumber=5 % Line numbers go in steps of 5
}

% Creates a new command to include a Perl script, the first parameter is the filename of the script (without .pl), the second parameter is the caption
\newcommand{\script}[2]{
\begin{itemize}
\item[]\lstinputlisting[caption=#2,label=#1]{#1}
\end{itemize}
}

%----------------------------------------------------------------------------------------
%	DOCUMENT STRUCTURE COMMANDS
%	Skip this unless you know what you're doing
%----------------------------------------------------------------------------------------

% Header and footer for when a page split occurs within a problem environment
\newcommand{\enterProblemHeader}[1]{
\nobreak\extramarks{#1}{#1 continued on next page\ldots}\nobreak
\nobreak\extramarks{#1 (continued)}{#1 continued on next page\ldots}\nobreak
}

% Header and footer for when a page split occurs between problem environments
\newcommand{\exitProblemHeader}[1]{
\nobreak\extramarks{#1 (continued)}{#1 continued on next page\ldots}\nobreak
\nobreak\extramarks{#1}{}\nobreak
}

\setcounter{secnumdepth}{0} % Removes default section numbers
\newcounter{homeworkProblemCounter} % Creates a counter to keep track of the number of problems

\newcommand{\homeworkProblemName}{}
\newenvironment{homeworkProblem}[1][Problem \arabic{homeworkProblemCounter}]{ % Makes a new environment called homeworkProblem which takes 1 argument (custom name) but the default is "Problem #"
\stepcounter{homeworkProblemCounter} % Increase counter for number of problems
\renewcommand{\homeworkProblemName}{#1} % Assign \homeworkProblemName the name of the problem
\section{\homeworkProblemName} % Make a section in the document with the custom problem count
\enterProblemHeader{\homeworkProblemName} % Header and footer within the environment
}{
\exitProblemHeader{\homeworkProblemName} % Header and footer after the environment
}

\newcommand{\problemAnswer}[1]{ % Defines the problem answer command with the content as the only argument
\noindent\framebox[\columnwidth][c]{\begin{minipage}{0.98\columnwidth}#1\end{minipage}} % Makes the box around the problem answer and puts the content inside
}

\newcommand{\homeworkSectionName}{}
\newenvironment{homeworkSection}[1]{ % New environment for sections within homework problems, takes 1 argument - the name of the section
\renewcommand{\homeworkSectionName}{#1} % Assign \homeworkSectionName to the name of the section from the environment argument
\subsection{\homeworkSectionName} % Make a subsection with the custom name of the subsection
\enterProblemHeader{\homeworkProblemName\ [\homeworkSectionName]} % Header and footer within the environment
}{
\enterProblemHeader{\homeworkProblemName} % Header and footer after the environment
}

%----------------------------------------------------------------------------------------
%	NAME AND CLASS SECTION
%----------------------------------------------------------------------------------------

\newcommand{\hmwkTitle}{Assignment 5} % Assignment title
%\newcommand{\hmwkDueDate}{} % Due date
\newcommand{\hmwkClass}{CS 362} % Course/class
%\newcommand{\hmwkClassTime}{} % Class/lecture time
%\newcommand{\hmwkClassInstructor}{} % Teacher/lecturer
\newcommand{\hmwkAuthorName}{Chewie Lin (Shuiqiang)} % Your name

%----------------------------------------------------------------------------------------
%	TITLE PAGE
%----------------------------------------------------------------------------------------
\begin{comment}

\title{
\vspace{2in}
\textmd{\textbf{\hmwkClass:\ \hmwkTitle}}\\
\normalsize\vspace{0.1in}\small{Due\ on\ \hmwkDueDate}\\
\vspace{0.1in}\large{\textit{\hmwkClassInstructor\ \hmwkClassTime}}
\vspace{3in}
}

\author{\textbf{\hmwkAuthorName}}
\date{} % Insert date here if you want it to appear below your name

%----------------------------------------------------------------------------------------

\begin{document}

\maketitle

%----------------------------------------------------------------------------------------
%	TABLE OF CONTENTS
%----------------------------------------------------------------------------------------

%\setcounter{tocdepth}{1} % Uncomment this line if you don't want subsections listed in the ToC

\newpage
\tableofcontents
\newpage
\end{comment}

\begin{document}

\homeworkSection{Re-factoring}
Using my team's tauqirs' dominion code, I didn't have to change the test itself
to run. However, there are many calls that will break if the array structure or
data structure change. \\

For instance, to set up the test for mine card, I have to manually set the hand
cards and deck cards in order to run the tests. If tarqirs change the struct or
how the array in the hand or deck are set up, the test would not be able to run. 
This is also the case when I'm assert and check if the test pass or fail. \\

The best way is to create some interface calls to change the hand card or deck
card through some function calls so that the test program does not need to know
about the data structure used. It's better to use something like the cardEffect
wrapper call which doesn't need to change. \\

\homeworkSection{Bug-Reports}
\texttt{Smithy}:
The first bug filed is smithy. The code failed the test to see whether the
player has drawn 3 new cards to his hand.
For example, with a full hand, after drawing 3 cards and discard the smithy card, the player
should end up with 7 cards, instead, it shows 6 cards only. The program did not
crash or return with errors so it could be some sort index error and return
prematurely. After checking out code, it appears that the for loop condition
should be 3 and not 2. \\

\lstinputlisting{smithybug.txt}
\lstinputlisting{BugsInTeammateCode.txt}


\texttt{mine}:
This was an interesting bug because it passes some test but failed at others. It
appears that when you trade copper or silver for higher value card, it is ok.
But whenever it is trade for something of equal value or less, it fails. At
first, it looks like maybe that's what the card should do but the official rule
does not block trading for equal or less value card. This could be due to some
bug when comparing the two card values. In the source code, it appears that the
second cannot be less than the first card or it will fail. 

\lstinputlisting{minebug.txt}
\lstinputlisting{BugsInTeammateCode2.txt}


\homeworkSection{Debugging}
\texttt{smithy}:
In my own dominion code, I'm  debuggint the smithy call as well. The test for this one
is simple but the error is different from my teammate. 
This bug appears not to draw no new card at all, Instead of expected hand count
of 7, i have only 4 cards in hand. 

\lstinputlisting{smithy-bugfix.txt}

In order to debug this, I have created several breakpoints and watchpoints. By
doing that I can see if how the hand cards change. Other breakpoints include
cardEffect and play\_smithy, watch for draw card loop index. 

\lstinputlisting{bk.txt}

In the case, I can see that
the smithy was played and discarded  when play\_smithy begins by playing 
the smithy card at handPos=0. The handCount watch point was reflect the new
change. But it didn't show any change to the hand from drawing new card.\\

After rerun the debugger, I have also set watchpoints for index i value. If the
loop stops this would help show me why. At first the i integer set to 0 as
expected but it did not step into the loop, instead, it proceeds to discarding
the smithy card. Looking at line 697 of the code, it becomes clear that the loop
condition is wrong and it exits too soon. \\

\lstinputlisting{debug.txt}

After fixing the loop condition, rerunning the test shows that it has passed the
tests. With the debugger, I can see that it has stepped through the loop this
time, adding a new card to hand each time. \\

\lstinputlisting{debug2.txt}





\begin{comment}
\lstinputlisting[language={C}]{refactor.txt}
\begin{SCfigure}[2.0][h]
\caption{
	It's might be better smithy, it gives you 2 cards and an +action so you
	can attack your opponent or improve your deck or chain it for a large
	hand.
}
\includegraphics[scale=0.5]{laboratory}
\end{SCfigure}
%------------------------------------------
%	PROBLEM 1
%-----------------------------------------
% To have just one problem per page, simply put a \clearpage after each problem
%----------------------------------------
\begin{homeworkProblem}
	1. List 3 different protocols that appear in the protocol column in 
	the unfiltered packet-listing window in step 7 above.
\lstinputlisting{summary.txt}
There are tcp, http, and tls protocols in the unfiltered packet.
\end{homeworkProblem}

%	PROBLEM 2
\begin{homeworkProblem}
	2. How long did it take from when the HTTP GET message was sent until the HTTP
OK reply was received? 
	\begin{verbatim}
The request was sent at 0.08 and returns at 0.17, the time taken is 0.09
	seconds.
	\end{verbatim}

\end{homeworkProblem}


%Problem 3
\begin{homeworkProblem}
	3. What is the Internet address of the gaia.cs.umass.edu (also known as www-
net.cs.umass.edu)?  What is the Internet address of your computer?

	\texttt{ gaia.cs.umass.edu = 128.119.245.12}\\

	\texttt{ source ip = 172.31.98.135}

\end{homeworkProblem}

%Problem 4
\begin{homeworkProblem}
	4. Screenshot the two HTTP messages (GET and OK) referred to in question 2
above. Make sure to include all pertinent information in the screenshot (Time
field, Internet addresses, etc). Paste these screenshots into your lab report.
	\lstinputlisting[language={}]{http.txt}
	This is the terminal output of tshark (terminal variant of wireshark)

\end{homeworkProblem}



%Problem 5
\begin{homeworkProblem}
\end{homeworkProblem}


	\begin{align}
	\begin{split}
		T(n) =
		\begin{cases}
			c_1 & \text{if } n=1 \\
			T(2\frac{n}{2}) + c_2n & \text{if } otherwise \\
		\end{cases}
	\end{split}
	\end{align}

	\renewcommand{\labelenumi}{\alph{enumi}.}
	\begin{enumerate}
	\end{enumerate}

%\includegraphics[width=1\columnwidth]{prob7} 

%matrix
\begin{align}                                                                  
A =                                                                            
\begin{bmatrix}                                                                
A_{11} & A_{21} \\                                                             
A_{21} & A_{22}                                                                
\end{bmatrix}                                                                  
\end{align}                                                                    


%equations
\begin{align}                                                                  
\begin{split}                                                                  
(x+y)^3   &= (x+y)^2(x+y)\\                                                    
&=(x^2+2xy+y^2)(x+y)\\                                                         
&=(x^3+2x^2y+xy^2) + (x^2y+2xy^2+y^3)\\                                        
&=x^3+3x^2y+3xy^2+y^3                                                          
\end{split}                                                                    
\end{align}

% import code file
\lstinputlisting{2color.txt}
\lstinputlisting[language=Octave]{BitXorMatrix.m}
\end{comment}

\end{document}
